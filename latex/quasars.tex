\documentclass[oneside,10pt]{article}
\usepackage{amsmath}
\usepackage{graphicx}

\usepackage[a4paper,margin=2cm,footskip=1cm]{geometry}

%\setlength\parindent{0pt}
%\usepackage{indentfirst}

\providecommand{\eqn}[1]{eqn.~(\ref{eqn:#1})}
\providecommand{\tab}[1]{Table~\ref{tab:#1}}
\providecommand{\fig}[1]{Figure~\ref{fig:#1}}

\providecommand{\var}{\mathrm{var}}

\providecommand{\lambdaobs}{\lambda^o}
\providecommand{\lambdarest}{\lambda^r}

\providecommand{\fitprefix}{random-n10000}

\title{A Linearized Quasar Spectrum Analysis}
\author{Daniel Margala}
\date{\today}

\begin{document}
\maketitle

\abstract{
We describe a method for performing simultaneous least squares fits of BOSS quasar 
spectra to a universal quasar continuum model with redshift dependent Ly$\alpha$ absorption, 
an observed frame transmission model, and individual quasar parameters. The goal is to develop 
a framework where observed frame systematics can be modeled and fit simultaneously with quasar continua. 
The resulting continua predications can be used to measure the density field of Ly$\alpha$ absorbers 
for BAO measurements.
}

\section{Sample Selection}

\subsection{BOSS Quasars}

We use BOSS DR12 quasars (\verb+OBJTYPE='QSO'+) with redshift greater than 0.5 (and \verb+ZWARNING>0+). For now, we only use  
10,000 quasars selected at random out of roughly 250,000 in this sample. This selection includes BALs and DLAs.
The redshift and median signal to noise distributions for our sample are shown in \fig{redshiftDistribution} and \fig{snDistribution}, respectively.   

\begin{figure}
\includegraphics[width=84mm]{fig/\fitprefix-redshift}
\centering
\caption{Redshift distribution of BOSS DR12 quasars selected for this analysis.}
\label{fig:redshiftDistribution}
\end{figure}

\begin{figure}
\includegraphics[width=84mm]{fig/\fitprefix-sn}
\centering
\caption{Median signal-to-noise distribution of BOSS DR12 quasars selected for this analysis.}
\label{fig:snDistribution}
\end{figure}

\subsection{Note On Data Processing}

We use the co-added spectrum for each quasar and ignore pixels where \verb+ivar == 0 || ANDMASK > 0+. 

It should be relatively straightforward to extend the framework of this method to make use of individual exposures. 

Are there exposure/observation/target parameters that we should include (e.g. alt, seeing, mags)? Perhaps the known 
magnitudes can be used instead of spectral tilt and amplitude model parameters.

\section{Linear Model}

\subsection{Observed Quasar Spectrum}

We model a quasar's rest frame flux $f_{rest}$ at rest wavelength $\lambdarest$ using

\begin{equation}
f_{rest}(\lambdarest) = A \left(\lambdarest/\lambda_{\star}\right)^{\nu} C(\lambdarest)
\label{eqn:restflux}
\end{equation}

where $A$ is a normalization constant, $\nu$ is a spectral tilt parameter about some pivot wavelength $\lambda_\star$, 
and $C(\lambdarest)$ is a universal quasar continuum model. The flux from the quasar, at redshift $z$, arriving at our galaxy
 is represented by $ f_{gal}(\lambdaobs)$ and can be expressed as

\begin{equation}
f_{gal}(\lambdaobs) = f_{rest}(\lambdarest) e^{-\tau(\lambdarest)} (1+z)^{-1}
\label{eqn:galflux}
\end{equation}

where $\lambdaobs = \lambdarest(1+z)$ is the observed frame frame wavelength and $\tau(\lambdarest)$ encapsulates the 
optical depth along the line of sight. The observed flux $f_{obs}(\lambdaobs)$ is then modeled as

\begin{equation}
f_{obs}(\lambdaobs) = f_{gal}(\lambdaobs) T(\lambdaobs)
\label{eqn:obsflux}
\end{equation}

where the function $T(\lambdaobs)$ is the observed frame transmission function. 

Using \eqn{restflux} and \eqn{galflux}, we can rewrite \eqn{obsflux} as:

\begin{equation}
f_{obs}(\lambdaobs) = A \left(\lambdarest/\lambda_{\star}\right)^{\nu} C(\lambdarest) e^{-\tau(\lambdarest)} (1+z)^{-1} T(\lambdaobs) 
\label{eqn:fluxmodel}
\end{equation}

We assume a power law model for redshift evolution of $\tau$:
\begin{equation}
\tau(\lambdarest) = a(\lambdarest) (1+z)^{b}
\end{equation}
where $b$ is a constant.

Now, \eqn{fluxmodel} can be linearized by taking the $\log$:
\begin{equation}
\log f_{obs}(\lambdaobs) = \log A + \nu \log \lambdarest/\lambda_\star + \log C(\lambdarest) - a(\lambdarest) (1+z)^{b} - \log (1+z) + \log T(\lambdaobs)
\label{eqn:linearmodel}
\end{equation}

In practice, there is also an additive random error that is almost certainly non-Gaussian after transforming:

\begin{equation}
\log (f + \delta f) \equiv \log f(1+\epsilon) \equiv \log f + \log (1+\epsilon) 
\end{equation}

assuming $\delta f$ is a Gaussian error of $f$ and $\epsilon = \delta f/f$ is the relative error in the flux measurement.

\subsection{Discretization}

Adding quasar index labels $i$ and pixel index labels $j$, \eqn{linearmodel} becomes:
\begin{equation}
\log f_{obs,i}(\lambdaobs_j) = \log A_i + \nu_i \log \lambdarest_{ij}/\lambda_\star + \log C(\lambdarest_{ij}) - a(\lambdarest_{ij}) (1+z_i)^{b} - \log (1+z_i) + \log T(\lambdaobs_j)
\end{equation}
with
\begin{equation}
\lambdarest_{ij} = \lambdaobs_j/(1+z_i) \; .
\end{equation}

%F_{ij} = f_i(\lambdaobs_j)(1+\epsilon_{ij}) \\
%\epsilon_{ij} = \delta F_{ij}/F_{ij} \\
%\tau_{ij} = a(\lambdarest_{ij}) (1+z_i)^{b} \\

We assume the following interpolation representation for $T$, $C$, and $a$,
\begin{gather}
T(\lambda) = \sum\limits_{p} w_p(\lambda) T_p \; , \;
C(\lambda) = \sum\limits_{q} w_q(\lambda) C_q  \; , \;
a(\lambda) = \sum\limits_{r} w_r(\lambda) a_r \; .
\end{gather}
The indices $p$, $q$ and $r$ correspond to parameters of the transmission, continuum and absorption models respectively. For linear interpolation, $w(\lambda)$ is a Kronecker delta function.

Therefore, the model is represented by $M = 2I + P + Q + R$ total parameters, where the quantities on the right hand side respectively represent
the total number of individual quasar parameters, the total number of transmission model parameters, 
the total number of continuum model parameters, and the total number of absorption model parameters.

\begin{figure}
\includegraphics[width=168mm]{fig/\fitprefix-matrix}
\centering
\caption{Example model matrix $X$. The transmission model parameter block is shown in blue. The continuum model parameter 
block is shown in green. The absorption model parameter block is shown in purple. The individual quasar parameter block, 
which includes both amplitude and spectral tilt parameters, is shown in orange. The dark green region indicates the 
range where the absorption model overlaps the continuum model.}
\label{fig:modelMatrix}
\end{figure}

\section{Least Squares}

For $N$ observations, $y$, with covariance $C$, and $M$ model parameters, $\beta$, the least squares solution to the overdetermined ($N > M$) set of linear equations, 
\begin{equation}
X \beta = y \; ,
\label{eqn:syseqs}
\end{equation}
is given by
\begin{equation}
\hat \beta = [X^T C^{-1} X]^{-1} [X^T C^{-1} y] \; ,
\end{equation}
where the matrix $X$ encodes the structure of the linear model. The matrix $X$ is $N$ by $M$, where $N$ is the total number 
of quasar pixels and $M$ is the total number of model parameters. This matrix is sparse; there are less than 5$N$ nonzero
entries. A truncated model matrix, selecting rows corresponding to 5 random quasars, is shown in \fig{modelMatrix}. 
Low-z quasars help constrain the transmission model at the blue end.

The estimate, $\hat\beta$, minimizes the $\chi^2$ objective function,
\begin{equation}
\chi^2 = [ y - X \beta ]^T C^{-1} [ y - X \beta ]  \; .
\end{equation}

We use the \verb+linear_model+ solver from the \verb+scikit-learn+ python package to estimate $\hat\beta$.

\subsection{Weighted Fit}

Assuming that the covariance matrix is diagonal (uncertainties are uncorrelated), we perform a weighted 
least squares fit by substituting $X = \sqrt{C^{-1}} \; \tilde{X}$ and $y = \sqrt{C^{-1}} \; \tilde{y}$, 
where the tilde represents the unweighted quantity. 

For now, we assume equal weights but we plan to use the actual pixel variances, although, care must 
be taken since the errors are non-Gaussian after linearizing the model by taking the log.

%We find a relationship between the observed flux uncertainty and our model quantity uncertainty to be
%
%\begin{equation}
%\var(\log(1+\epsilon_{ij})) \approx \var(f_{ij})^2 / f_{ij}^4 \; ,
%\end{equation}
%
%by performing a numerical study.

\subsection{Model Parameters}

A natural choice for the transmission model parameter grid is the BOSS observed frame wavelength grid. 
We use a linearly spaced grid for the continuum model parameters. The absorption model is naturally defined using rest frame 
wavelengths therefore we use the same grid that is used for the continuum model. The absorption power law index
is set to the value from FG08, $b = 3.92$.
The spectral tilt reference wavelength $\lambda_\star$ is set to where the continuum is constrained to be 1.

\subsection{Parameter Constraints}

Parameter constraints are included as additional equations in the linear model. This amounts to appending rows to $X$ and $y$. 
The constraints are enforced by assigning relatively high weights to these entries. Alternatively, they could be enforced using
regularization (haven't fully explored this option yet).

The continuum model is constrained by setting the geometric mean to be 1 in a specified wavelength range via 
\begin{equation}
N_q^{-1} \sum_q \log C_q =  0 \; .
%N_q^{-1} \sum_q \log C_q \equiv \log \left(\prod_q C_q\right)^{1/N_q} = \log 1 \equiv 0 \; .
\end{equation}
Each of the transmission model parameters are independently constrained to be 1 via
\begin{equation}
\log T_p = 0 \; .
\end{equation}
The mean of the spectral tilt indices is constrained to be 0 via
\begin{equation}
N_i^{-1} \sum_i \nu_i = 0 \; .
\end{equation}

\section{Results}

The continuum model estimate is shown in \fig{continuum}. 

The transmission model estimate is shown in \fig{transmission}.

The absorption model estimate is shown in \fig{absorption}. The absorption coefficients are in the ballpark 
of (Faucher-Giguere, 2008) and decrease significantly past Ly$\alpha$ emission wavelength. 
(Try refitting with mean absorption coefficient fixed for $\lambdarest < \lambda_{Ly\alpha}$.)

A few randomly selected example quasar spectra and their predicted unabsorbed continuum are shown in \fig{examples}. 

\clearpage

\begin{figure}
\includegraphics[width=168mm]{fig/\fitprefix-continuum}
\centering
\caption{The estimated continuum model is shown in black. The mean of the continuum model is 
constrained to be 1 in the gray region. The orange vertical lines indicate known quasar emission lines.}
\label{fig:continuum}
\end{figure}

\begin{figure}
\includegraphics[width=168mm]{fig/\fitprefix-transmission}
\centering
\caption{The estimated transmission model is shown in black. The green vertical lines indicate the Ballmer spectral series.
The magenta vertical lines indicate known sky lines.}
\label{fig:transmission}
\end{figure}

\clearpage

\begin{figure}
\includegraphics[width=126mm]{fig/\fitprefix-absorption}
\centering
\caption{The coefficients $a(\lambdarest)$ for the absorption model $\tau(\lambdarest) = a(\lambdarest)(1+z)^b$ with constant $b=3.92$ are shown in black. The dashed horizontal line indicates the absorption coefficient 
value $a=0.0018$ from (Faucher-Giguere, 2008). The orange vertical lines indicate known quasar emission lines. }
\label{fig:absorption}
\end{figure}

%\begin{figure}
%\includegraphics[width=84mm]{fig/\fitprefix-nuVsA}
%\centering
%\caption{Individual quasar spectral tilt vs amplitude parameters. The points are colored by the quasar's redshift.}
%\label{fig:nuVsA}
%\end{figure}

\begin{figure}
\includegraphics[width=168mm]{fig/\fitprefix-examples}
\caption{The observed combined spectrum for a target is shown in blue. The unabsorbed continuum prediction for the target is shown in red.
The orange vertical lines indicate known quasar emission lines. }
\label{fig:examples}
\end{figure}


\end{document}  
